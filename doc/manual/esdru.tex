
%=============================================================================
% ... THIS IS chapter{ESDRU: Energy-based Spatial Distortion Reference Unit} ...
% Revisions
% Dec.2021 - First version
%=============================================================================
%=============================================================================
\chapter{ESDRU: Energy-based Spatial Distortion Reference Unit}
%=============================================================================

\section{Introduction}

Subjective listening tests benefit from having known quality references within
the listening test material. For P.800 monaural listening tests, the MNRU has
been widely adopted. Another example is BS.1536, where the anchor condition(s)
are formed using a low pass filter. For testing stereo conditions, e.g. within
P.811, it is beneficial to have the reference condition include the stereo or 
spatial dimension. P.811 outlines two variants of a spatial distortion reference
unit (SDRU). Here, the ESDRU variant is described.



\section{Description of the Algorithm}

An overview of the algorithm can be found in P.811 \cite{P.811}. The basis for
the modulation is the same as for the SDRU \cite{SDRU}, and can be described by
the equation

  \[
    \left\{
       \begin{array}{ll}
         {SDRU}_L(n) = g_4(n) \left( \alpha Ch_L(n)+(1-\alpha)Ch_R(n) \right) \\
         {SDRU}_R(n) = g_5(n) \left( \alpha Ch_R(n)+(1-\alpha)Ch_L(n) \right) \\
       \end{array}
     \right.
  \]

where $n$ is the sample index and $\alpha$ is the spatial degradation factor that
ranges between $[0,1.0]$. $g_4(n)$ and $g_5(n)$ are complementary gains from a
spatial modulation function where
 
  \[
    g_5(n) = 1 + \alpha - g_4(n),
  \]
  
meaning that $g_5(n)$ will be $1$ when $g_4(n)$ is $\alpha$ and $g_5(n)$ will be $\alpha$
 when $g_4(n)$ is $1$. The modulation function is what differs between the ESDRU and
\cite{SDRU}. For the latter, a triangular modulation function with a period of 1 second
is used. Here, the modulation is characterized by a step-wise random pattern. The function
that generates the modulation pattern is called g\_mod\_nrg.

{\tt\small
\begin{verbatim}
void g_mod_nrg(
    const double *input, /* i: Stereo input signal                 */
    const long length,   /* i: Length of input signal in samples   */
    const long step,     /* i: Length of transition in samples     */
    const double e_step, /* i: Energy step in high energy segments */
          double *m      /* o: Modulation curve                    */
)
\end{verbatim}
}

The \texttt{input} is the interleaved stereo signal read from file, \texttt{length} is the number
of samples, \texttt{step} is the interval where new values of the modulation is considered and also
the number of samples in the transition and \texttt{e\_step} is the allowed step size during high-energy
segments. High-energy segments of the input signal are defined as when the short-term energy estimate 
$e_s(n)$ is above the long-term energy estimate $e_l(n)$. The energy of the input signal is first computed
by the function \texttt{energy}.

{\tt\small
\begin{verbatim}
/*-------------------------------------------------
 * Compute energy of left + right
 * e( n ) = left( n ).^2 + right( n ).^2
 *-------------------------------------------------*/
void energy(
    const double *input,  /* i: Input signal       */
    double *e,            /* i: Output signal      */
    const long length     /* i: Length of signal   */
)
{
    long i;
    for( i = 0; i < length; i++ )
    {
        e[i] = input[2 * i] * input[2 * i] + input[2 * i + 1] * input[2 * i + 1];
    }

    return;
} 
\end{verbatim}
}

The short-term and short-term energy envelopes are derived by applying a one-pole first order IIR-filter
of the form:

  \[
    y(n) = \beta x(n) + (1 - \beta) y(n-1)
  \]

where $\beta$ is the low-pass filtering factor. The filtering is applied in forward and reversed time order
on for both $e_s(n)$ and $e_l(n)$ to get a symmetric filtering of the energy values. This is implemented using
the function \texttt{ar1}, where the last argument controls forward (1) or backward (-1) time order. Finally,
a scaling of 0.77813 is applied to the long-term envelope $e_l(n)$ to lower it relative to short-term envelope 
$e_s(n)$. 

{\tt\small
\begin{verbatim}
    ar1( 0.001, e, es, length, -1 );
    ar1( 0.001, es, es, length, 1 );
    ar1( 0.0001, es, el, length, -1 );
    ar1( 0.0001, el, el, length, 1 );
    scale( el, 0.77813, el, length );
\end{verbatim}
}

The same operation may also be written in MATLAB code as follows.

{\tt\small
\begin{verbatim}
% Short-term energy envelope
alpha1 = 0.001;
[B,A] = iir1(alpha1);
es = filter(B,A,flipud(filter(B,A,flipud(e))));

% Long-term energy envelope
alpha2 = 0.0001;
beta = 0.77813;
[B,A] = iir1(alpha2);
el = beta*filter(B,A,flipud(filter(B,A,flipud(es))));
\end{verbatim}
}

The modulation may change at a step size of 1.5 ms. At each change point, a new value is chosen
with a probility of 0.2. This is implemented by comparing a random number generator with 0.2. If the decision
is taken to change level, the next level is generated as a random number in the range $[0,1.0]$. Next, if
the current signal segment is considered to be a high-energy segment, the step towards the new random level is
scaled with \texttt{e\_step}, effectively limiting the step-size during high-energy segments. By default, 
\texttt{e\_step} is set to 0.5, which allows to go half-way to the new random level. If \texttt{e\_step} would
be set to zero, the modulation function would not change at all during high energy segments. Finally, if \texttt{e\_step}
would be set to 1, the step size would not be limited at all during high energy segments.

Once a new random value has been set, a cosine-shaped transition window is used to make a smooth transition to the
new value. The following code generates a modulation function $g\_mod\_nrg(n)$ in the range $[0,1.0]$. 

{\tt\small
\begin{verbatim}
    m_prev = 1.0;
    while( i < length )    
    {
        if( (rand() / ((double)RAND_MAX)) < 0.2 )
        { 
            if( es[i] < el[i] )
            {
                m_delta = 1.0;
            }
            else
            {
                m_delta = e_step;
            }
            m_new = rand() / ((double)RAND_MAX) * m_delta + m_prev * (1.0 - m_delta);
        }
        else
        {
            m_new = m_prev;
        }

        for(j = 0; j < step && i < length; i++,j++ )
        {
            xf_win = 0.5 * (1.0 - cos( LOCAL_PI * j / step ));
            m[i] = m_new * xf_win + m_prev * (1.0 - xf_win);
        }
        m_prev = m_new;
    }
\end{verbatim}
}    
    
The modulation functions $g_4(n)$ and $g_5(n)$ may then be derived from $g\_mod\_nrg(n)$ as 

  \[
    \left\{
       \begin{array}{ll}
            g_4(n) = 1.0 - g\_mod\_nrg(n) (1.0 - \alpha) \\
            g_5(n) = 1.0 + \alpha - g_4(n)               \\
       \end{array}
     \right.
  \]
  
Once the modulation function has been generated, the spatial distortion is applied using the 
function \texttt{apply\_spatial\_dist}.


\begin{figure}[htp]
    \begin{center}
        \includegraphics[scale=1.0]{esdru_spatial_modulation.pdf}
  \end{center}
  \caption{The energy of the input file is analyzed and then low-pass filtered to form a 
           short-term estimate $e_s(n)$ and a long-term estimate $e_l(n)$. When the short-term  
           estimate is above the long-term estimate, $e_s(n)>e_l(n)$, it is considered a high-energy
           segment and the change in modulation is limited by $e_{step}$.
           \label{fig:esdru_spatial_modulation} }
\end{figure}



\section{Usage of esdru.exe}

{\tt\small
\begin{verbatim}
esdru.exe [options] <alpha> <input file> <output file>

<alpha>           Alpha value [0.0 ... 1.0]
<input file>      Input file, 16 bit Stereo PCM
<output file>     Output file, 16 bit Stereo PCM

Options:
-sf FS            Sampling frequency FS Hz (Default: 48000 Hz)
-e_step S         Max step S during high energy [0.0 ... 1.0] (Default: 0.5)
-seed I           Set random seed I [unsigned int] (Default: 1)
\end{verbatim}
}